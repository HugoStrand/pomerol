\documentclass{article}
\usepackage{amsmath}
\usepackage{amssymb}

\oddsidemargin 0.0cm
\textwidth 7.0in 
\textheight 9.0in 
\topmargin 0.0in 

\newcommand{\up}{\ensuremath{\uparrow}}
\newcommand{\down}{\ensuremath{\downarrow}}
\newcommand{\bra}[1]{\ensuremath{\langle#1|}}
\newcommand{\ket}[1]{\ensuremath{|#1\rangle}}
\newcommand{\melem}[3]{\ensuremath{\langle #1|#2|#3\rangle}}
\newcommand{\aver}[1]{\ensuremath{\langle#1\rangle}}
\newcommand{\bc}{\ensuremath{\bar c}}
\newcommand{\hc}{\ensuremath{\hat c}}
\newcommand{\hcd}{\ensuremath{\hat c^\dagger}}
 
\DeclareMathOperator*{\sgn}{sgn}
\DeclareMathOperator{\Sp}{Sp}
\DeclareMathOperator*{\Tr}{Tr}

\begin{document}

\section{A short note on symmetry analysis}

Consider a general Hamiltonian of a many-body fermion system with $N$ fermionic modes and
pair interactions:
\begin{equation}\label{Hamiltonian}
    \hat H = \sum_{\alpha\beta} h_{\alpha\beta}\hcd_\alpha \hc_\beta +
        \frac{1}{2}\sum_{\alpha\beta\gamma\delta} U_{\alpha\beta\gamma\delta}
        \hcd_\alpha \hcd_\beta \hc_\gamma \hc_\delta
\end{equation}

\paragraph{Theorem 1.} If there exists a pair of permutation $N\times N$ matrices $P_c$ and $P_a$,
obeying additional conditions $(P_c)_{jl}=(P_c)_{lj}=1$ and $(P_a)_{ik}=(P_a)_{ki}=1$, such that
\begin{equation}\label{theorem1:eq_h}
    h_{\alpha\beta} = \sum_{\alpha'\beta'}(P_c)_{\alpha\alpha'} h_{\alpha'\beta'} (P_a)_{\beta'\beta},\\
\end{equation}
\begin{equation}\label{theorem:eq_U}
    U_{\alpha\beta\gamma\delta} = \sum_{\alpha'\beta'\gamma'\delta'}
        (P_c)_{\alpha\alpha'}(P_c)_{\beta\beta'}
        U_{\alpha'\beta'\gamma'\delta'}
        (P_a)_{\gamma'\gamma}(P_a)_{\delta'\delta},
\end{equation}
then
\begin{equation}\label{theorem1:result}
    g_{ij}(\omega) = g_{kl}(\omega).
\end{equation}

\paragraph{Proof.} Equalities \ref{theorem1:eq_h}, \ref{theorem:eq_U} mean that the action
corresponding to Hamiltonian \ref{Hamiltonian} is invariant under a permutation (relabeling) $R_c$
of variables $\bc_\alpha$ and a permutation $R_a$ of variables $c_\beta$.
The additional conditions $(P_c)_{jl}=(P_c)_{lj}=1$ and $(P_a)_{ik}=(P_a)_{ki}=1$ make
permutation $P_c$ swap $\bc_j$ with $\bc_l$ and permutation $P_a$ swap $c_i$ with $c_k$.
It is then clear that definitions of $g_{ij}(\omega)$ and $g_{kl}(\omega)$ coincide up
to the described relabeling. \textbf{Q.E.D.}

\vspace{5mm}
\textbf{TODO:} Is there an efficient algorithm that allows to check the existence of $P_c$ and $P_a$?

\end{document}
